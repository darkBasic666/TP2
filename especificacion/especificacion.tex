% Clase y configuracion de tipo de documento
\documentclass[10pt,a4paper,spanish]{article}
% Inclusion de paquetes
\usepackage{a4wide}
\usepackage{amsmath, amscd, amssymb, amsthm, latexsym}
\usepackage[spanish]{babel}
\usepackage[utf8]{inputenc}
\usepackage[width=15.5cm, left=3cm, top=2.5cm, height= 24.5cm]{geometry}
\usepackage{fancyhdr}
\pagestyle{fancyplain}
\usepackage{listings}
\usepackage{enumerate}
\usepackage{xspace}
\usepackage{longtable}
\usepackage{caratula}
% incluye macros espec materia
%practicas
\newcommand{\practica}[2]{%
    \title{Pr\'actica #1 \\ #2}
    \author{Algoritmos y Estructura de Datos I}
    \date{Primer Cuatrimestre 2013}

    \maketitle
}

%ejercicios
\newtheorem{exercise}{Ejercicio}
\newenvironment{ejercicio}{\begin{exercise}\rm}{\end{exercise} \vspace{0.2cm}}
\newenvironment{items}{\begin{enumerate}[i)]}{\end{enumerate}}
\newenvironment{subitems}{\begin{enumerate}[a)]}{\end{enumerate}}
\newcommand{\sugerencia}[1]{\noindent \textbf{Sugerencia:} #1}

%tipos basicos
\newcommand{\rea}{\mathsf{Float}}
\newcommand{\float}{\mathbb{R}}
\newcommand{\ent}{\mathbb{Z}}
\newcommand{\cha}{\mathsf{Char}}
\newcommand{\bool}{\mathsf{Bool}}
\newcommand{\Then}{\rightarrow}
\newcommand{\Iff}{\leftrightarrow}

\newcommand{\mcd}{\mathrm{mcd}}
\newcommand{\prm}[1]{\ensuremath{\mathsf{prm}(#1)}}
\newcommand{\sgd}[1]{\ensuremath{\mathsf{sgd}(#1)}}


%listas
\newcommand{\TLista}[1]{[#1]}
\newcommand{\lvacia}{\ensuremath{[\ ]}}
\newcommand{\longitud}[1]{\left| #1 \right|}
\newcommand{\cons}[1]{\mathsf{cons}(#1)}
%\newcommand{\cons}[1]{\mathrm{cons}(#1)}
\newcommand{\indice}[1]{\mathsf{indice}(#1)}
%\newcommand{\indice}[1]{\mathrm{indice}(#1)}
\newcommand{\conc}[1]{\mathsf{conc}(#1)}
%\newcommand{\conc}[1]{\mathrm{conc}(#1)}
\newcommand{\cab}[1]{\mathsf{cab}(#1)}
%\newcommand{\cab}[1]{\mathrm{cab}(#1)}
\newcommand{\cola}[1]{\mathsf{cola}(#1)}
%\newcommand{\cola}[1]{\mathrm{cola}(#1)}
\newcommand{\sub}[1]{\mathsf{sub}(#1)}
%\newcommand{\sub}[1]{\mathrm{sub}(#1)}
\newcommand{\en}[1]{\mathsf{en}(#1)}
%\newcommand{\en}[1]{\mathrm{en}(#1)}
\newcommand{\cuenta}[2]{\mathsf{cuenta}\ensuremath{(#1, #2)}}
%\newcommand{\cuenta}[2]{\ensuremath{\mathrm{cuenta}(#1, #2)}}
\newcommand{\suma}[1]{\mathsf{suma}(#1)}
%\newcommand{\suma}[1]{\mathrm{suma}(#1)}
\newcommand{\twodots}{\mathrm{..}}

% Acumulador
\newcommand{\acum}[1]{\mathsf{acum}(#1)}

% \selector{variable}{dominio}
\newcommand{\selector}[2]{#1~\ensuremath{\leftarrow}~#2}
\newcommand{\selec}{\ensuremath{\leftarrow}}

% -----------------
% Especificacion
% -----------------

% Para problemas con resultado:
% \begin{problema}{nombre}{argumentos}{resultado}
%       \modifica{variables}
%       \requiere[nombre]{condicion}
%       \requiere[nombre]{condicion}
%       ...
%       \asegura[nombre]{condicion}
%       \asegura[nombre]{condicion}
%       ...
% \end{problema}

% Para problemas sin resultado:
% \begin{problema*}{nombre}{argumentos}
%       \modifica{variables}
%       \requiere[nombre]{condicion}
%       \requiere[nombre]{condicion}
%       ...
%       \asegura[nombre]{condicion}
%       \asegura[nombre]{condicion}
%       ...
% \end{problema*}

\newenvironment{problema}[3]{
    \vspace{0.2cm}
    \noindent \textsf{problema #1}\ensuremath{(#2) = #3\{}\\
    \begin{tabular}{p{0.02\textwidth} p{0.85 \textwidth}}
}{
    \end{tabular}

    \noindent \ensuremath{\}}
    \vspace{0.15cm}
}

\newenvironment{problema*}[2]{
    \vspace{0.2cm}
    \noindent \textsf{problema #1}\ensuremath{(#2)\{}\\
    \begin{tabular}{p{0.02\textwidth} p{0.85 \textwidth}}
}{
    \end{tabular}

    \noindent \ensuremath{\}}
    \vspace{0.15cm}
}

\newcommand{\requiere}[2][]{& \textsf{requiere #1: }\ensuremath{#2};\\}
\newcommand{\asegura}[2][]{& \textsf{asegura #1: }\ensuremath{#2};\\}
\newcommand{\modifica}[1]{& \textsf{modifica }\ensuremath{#1};\\}
\newcommand{\pre}[1]{\textsf{pre}\ensuremath{(#1)}}
\newcommand{\aux}[2]{& \textsf{aux }\ensuremath{#1 = #2};\\}
\newcommand{\problemanom}[1]{\textsf{#1}}
\newcommand{\problemail}[3]{\textsf{problema #1}\ensuremath{(#2) = #3}}
\newcommand{\problemailsinres}[2]{\textsf{problema #1}\ensuremath{(#2)}}
\newcommand{\requiereil}[2][]{\textsf{requiere #1: }\ensuremath{#2}}
\newcommand{\asegurail}[2][]{\textsf{asegura #1: }\ensuremath{#2}}
\newcommand{\modificail}[1]{\textsf{modifica }\ensuremath{#1}}
\newcommand{\auxil}[2]{\textsf{aux }\ensuremath{#1 = #2};}
\newcommand{\auxnom}[1]{\textsf{aux }\ensuremath{#1}}

% -----------------
% Tipos compuestos
% -----------------

\newcommand{\Pred}[1]{\mathit{#1}}
\newcommand{\TSet}[1]{\textsf{Conjunto}\ensuremath{\langle #1 \rangle}}
\newcommand{\TSetFinito}[1]{\textsf{Conjunto}\ensuremath{\langle #1 \rangle}}
\newcommand{\TRac}{\tiponom{Racional}}
\newcommand{\TVec}{\tiponom{Vector}}
\newcommand{\True}{\mathrm{True}}
\newcommand{\False}{\mathrm{False}}
\newcommand{\Func}[1]{\mathrm{#1}}
\newcommand{\cardinal}[1]{\left| #1 \right|}
\newcommand{\enum}[2]{\ensuremath{\mathsf{#1} = \langle \mathsf{#2} \rangle}}

\newenvironment{tipo}[1]{%
    \vspace{0.2cm}
    \textsf{tipo #1}\ensuremath{\{}\\
    \begin{tabular}[l]{p{0.02\textwidth} p{0.02\textwidth} p{0.82 \textwidth}}
}{%
    \end{tabular}

    \ensuremath{\}}
    \vspace{0.15cm}
}

\newcommand{\observador}[3]{%
    & \multicolumn{2}{p{0.85\textwidth}}{\textsf{observador #1}\ensuremath{(#2):#3}}\\%
}
\newcommand{\observadorconreq}[3]{
    & \multicolumn{2}{p{0.85\textwidth}}{\textsf{observador #1}\ensuremath{(#2):#3 \{}}\\
}
\newcommand{\observadorconreqfin}{
    & \multicolumn{2}{p{0.85\textwidth}}{\ensuremath{\}}}\\
}
\newcommand{\obsrequiere}[2][]{& & \textsf{requiere #1: }\ensuremath{#2};\\}

\newcommand{\explicacion}[1]{&& #1 \\}
\newcommand{\invariante}[1]{%
    & \multicolumn{2}{p{0.85\textwidth}}{\textsf{invariante }\ensuremath{#1}}\\%
}
\newcommand{\auxinvariante}[2]{
    & \multicolumn{2}{p{0.85\textwidth}}{\textsf{aux }\ensuremath{#1 = #2}};\\
}

\newcommand{\tiponom}[1]{\ensuremath{\mathsf{#1}}\xspace}
\newcommand{\obsnom}[1]{\ensuremath{\mathsf{#1}}}

% -----------------
% Ecuaciones de terminacion en funcional
% -----------------

\newenvironment{ecuaciones}{%
    $$
    \begin{array}{l @{\ /\ (} l @{,\ } l @{)\ =\ } l}
}{%
    \end{array}
    $$
}

\newcommand{\ecuacion}[4]{#1 & #2 & #3 & #4\\}

% Listas por comprension. El primer parametro es la expresion y el
% segundo tiene los selectores y las condiciones.
\newcommand{\comp}[2]{[\,#1\,|\,#2\,]}


% Encabezado
\lhead{Algoritmos y Estructuras de Datos I}
\rhead{Grupo 22}
% Pie de pagina
\renewcommand{\footrulewidth}{0.4pt}
\lfoot{Facultad de Ciencias Exactas y Naturales}
\rfoot{Universidad de Buenos Aires}

\begin{document}

% Datos de caratula
\materia{Algoritmos y Estructuras de Datos I}
\titulo{Trabajo Pr\'actico N\'umero 1}
%\subtitulo{}
\grupo{Grupo: 22}

\integrante{Maximiliano Leon Paz}{251/14}{m4xileon@gmail.com}
\integrante{Rosso, Sebasti\'an}{921/13}{sebi.rr@hotmail.com}
\integrante{Mena, Manuel}{313/14}{manuelmena1993@gmail.com}

\maketitle

\newpage

% Para crear un indice
%\tableofcontents

% Forzar salto de pagina
\clearpage



% Otro salto de pagina
% \newpage

\section{Problemas}

	
	\begin{ejercicio}
	
	Posiciones Más Oscuras:
	
		\begin{problema}{PosicionesMasOscuras}{i:Imangen}{res::\TLista{(\ent,\ent)}}
			\asegura{mismos(res,masOscuras(i))}
		\end{problema}
	\end{ejercicio}

	\begin{ejercicio}
	
	Top 10: 
	
		\begin{problema}{Top10}{g:Galeria}{res::\TLista{Imagen}}
			\requiere[Al menos hayan 10]{\longitud{imagenes(g)}\geq 10}
			\asegura[Tamaño]{\longitud{res} == 10}
			\asegura[Top]{\forall{(i\leftarrow[0 .. 9), mas \leftarrow[0 .. i)  ,menos \leftarrow [0 .. \longitud{imagenes(g)}) ,res_{mas} \neq imagenes(g)_{menos})} 
			\newline (res_i \leq res_{mas}) \land  (res_i \geq imagenes(g)_{mas}) }
		\end{problema}
	\end{ejercicio}

	\begin{ejercicio}
	
	La Más Chiquita Con Punto Blanco:
	
		\begin{problema}{laMasChiquitaConPuntoBlanco}{g:Galeria}{res:Imagen}
			\requiere[Imagen con un punto blanco]{\exists{(i \leftarrow imagenes(g))}
			\newline tengaUnPuntoBlanco(i) }
			\asegura[Pertenecia]{res \in lasMasChiquitas(\TLista{i|i \leftarrow imagenes(g) , 
			\newline tengaUnPuntoBlanco(i) }) }
		\end{problema}
	\end{ejercicio}

	\begin{ejercicio}
		
		Agregar imagen:
		
		\begin{problema}{agregarImagen}{g:Galeria, i:Imagen}{}
			\requiere[Nueva imagen]{i \notin imagenes(g)}
			\modifica{g}
			\asegura[Un elemento mas]{\longitud{imagenes(g)} == \longitud{imagenes(pre(g))} + 1}
			\asegura[Se agrega]{i \in imagenes(g)}
			\asegura[Otras imagenes se mantienen]{imagenesAnterioresPertenecen(g)}
			\asegura[0 votos]{votos(g, i) == 0}
			\asegura[Otros votos se mantienen]{otrosVotosIguales(g, i)}
		\end{problema}
	\end{ejercicio}
	
	\begin{ejercicio}
		
		Votar:
		
		\begin{problema}{votar}{g:Galeria, i:Imagen}{}
			\requiere[Pertenencia]{i \in imagenes(g)}
			\modifica{g}
			\asegura[Misma cantidad]{\longitud{imagenes(g)} == \longitud{imagenes(pre(g))}}
			\asegura[Mismas imagenes]{mismasImagenes(g)}
			\asegura[Un voto mas]{votos(g, i) == votos(pre(g), i) + 1}
			\asegura[Otros votos se mantienen]{otrosVotosIguales(g, i)}
		\end{problema}	
	\end{ejercicio}
	
	\begin{ejercicio}
		
		Eliminar mas votada:
		
		\begin{problema}{eliminarMasVotada}{g:Galeria}{}
			\requiere[Al menos una imagen]{\longitud{imagenes(g)} \geq 1}
			\modifica{g}
			\asegura[Se elimina una]{\longitud{imagenes(g)} == \longitud{imagenes(pre(g))} - 1}
			\asegura[Elimina mas votada]{eliminaMasVotada(g)}
		\end{problema}	
	\end{ejercicio}		

\section{Auxiliares}


	
%	\item \auxil{corte(i,j,n,m:\ent,img:\TLista{\TLista{(\ent,\ent,\ent)}}):\TLista{\TLista{(\ent,\ent,\ent)}}}{ armarImagen(\TLista{{img_x}_y | x \leftarrow [i*(\longitud{img} \ n) .. (i+1)*(\longitud{img} \ n)), y \leftarrow [j*(\longitud{img_0} \ m) .. (j+1)*(\longitud{img_0} \ m))},n,m)}
		

	
%	\item \auxil{armarImagen(lPx:\TLista{(\ent,\ent,\ent)},n,m:\ent):\TLista{\TLista{(\ent,\ent,\ent)}}}{\TLista{armarFila(lPx,i,j,n,m)|i \leftarrow [0..n), j \leftarrow [0 .. m)}}
	
% (falta terminar: armarFila)	
%	\item \auxil{armarFila(lPx: \TLista{(\ent,\ent,\ent)}):\TLista{(\ent,\ent,\ent)}}{

\subsection{Comunes}
\begin{itemize}
	\item \auxil{masOscuras(i:Imangen) : \TLista{(\ent,\ent)} }{
		\newline \TLista{(x,y)| x \leftarrow  [0..ancho(i)) , y \leftarrow [0 .. alto(i)) , \forall{(i \leftarrow  [0..ancho(i)) , j \leftarrow [0 .. alto(i))}
		\newline blancura(color(x,y))  \leq blancura(color(i,j)) }
	}
	\item \auxil{blancura(p:Pixel) : \ent }{
		red(p) + blue(p) + green(p)
	}
	\item \auxil{lasMasChiquitas(g:Galeria) : \TLista{Imagen} }{
		%\newline 
		\TLista{i|i \leftarrow imagenes(g), j \leftarrow imagenes(g), size(i) \leq size(j) }
	}
	\item \auxil{tengaUnPuntoBlanco(i:Imagen) : Bool }{
		\newline \exists{(x \leftarrow  [0..ancho(i)) , y \leftarrow [0 .. alto(i))}blanco(color(i,x,y)) 
		\newline \land \forall{(p\leftarrow pixelesAyacentes(i,x,y))} \neg blanco(p)
	}
	\item \auxil{blanco(p:Pixel) : Bool }{
		red(p) + blue(p) + green(p) == 3 * 255
	}
	\item \auxil{size(i:Imagen) : \ent }{
		ancho(i) + alto(i)
	}
	\item \auxil{pixelesAyacentes(img:Imagen,x,y:\ent) : \TLista{Pixel} }{
		\newline \TLista{color(img,i,j)| i \leftarrow [x-1 .. x+1], j \leftarrow [y-1 .. y + 1] , 
		\newline enRango(ancho(img),alto(img),1,i,j) \land (x \neq i) \land (y \neq j) }
	}
	\item \auxil{enRango(ancho,alto,r,x,y:\ent) : Bool}{
		\newline (x - r  \geq 0) \land (x + r  \leq ancho - 1) \land (y - r  \geq 0) \land (y + r \leq alto -1 ) 
	}
	
\end{itemize}

\subsection{Ejercicio 4}
\begin{itemize}
	\item \auxil{imagenesAnterioresPertenecen(g:Galeria) : Bool}{
		\newline (\forall img \leftarrow imagenes(pre(g)))img \in imagenes(g)
	}
	
	\item \auxil{votosAnterioresIguales(g:Galeria) : Bool}{
		\newline (\forall i \leftarrow [0..\longitud{imagenes(pre(g))}))votos(g, imagenes(pre(g))_i) == votos(g, imagenes(g)_i)
	}
\end{itemize}

\subsection{Ejercicio 5}
\begin{itemize}
	\item \auxil{mismasImagenes(g:Galeria) : Bool}{
		\newline (\forall i \leftarrow [0..\longitud{imagenes(g)}))imagenes(pre(g))_i == imagenes(g)_i
	}
\end{itemize}
	
\subsection{Ejercicio 6}
\begin{itemize}
	\item \auxil{eliminarMasVotada(g:Galeria) : Bool}{
		\newline (\exists img \leftarrow imagenes(pre(g)), masVotada(pre(g), img)) 
		\newline mismos(imagenes(g), filtrarImagen(imagenes(pre(g)), img))
	}		
	
	\item \auxil{masVotada(g:Galeria, img:Imagen) : Bool}{
		\newline (\forall i \leftarrow imagenes(g)) votos(g, img) \geq votos(g, i)
	}
	
	\item \auxil{filtrarImagen(lista:[Imagen], img:Imagen) : [Imagen]}{
		[i | i \leftarrow lista, i \neq img]
	}
\end{itemize}	


        
\end{document}
